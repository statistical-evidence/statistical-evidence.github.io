{Na atualização de 2007 (IBGE), as cidades brasileiras foram classificadas em cinco grandes níveis, a saber:}


\begin{itemize}
\item Metrópoles, representadas pelos 12 principais centros urbanos do país;
\item Capitais regionais, constituídas 70 grandes cidades que têm área de influência
de âmbito regional;
\item Centros sub-regionais, compostos por 169 cidades com atividades de gestão menos complexas;
\item Centros de zona, formados por 556 cidades de menor porte e com atuação restrita à sua área imediata; e
\item Centros locais, representados pelas demais 4.473 cidades cuja centralidade e atuação não extrapolam os limites do seu município, servindo apenas aos seus habitantes.
\end{itemize}
	

